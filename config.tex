%!TEX root = LaTeX/main.tex

%%%%%%%%%%%%%%%%%%%%%%%%%%%%%%%%%%%%%%%%%%
%%%%%%%%%%%%%%%%%%%%%%%%%%%%%%%%%%%%%%%%%%
%%
%%    S E M E S T E R P R O G R A M M
%%
%%    LaTeX-Vorlage:   Sebastian Fiedlschuster  E 06  (B-XX)
%%    <fiedlschuster@wingolf.org>
%%    Dez 2011
%%
%%    Zur Erstellung eines Semesterprogrammes diese Datei anpassen
%%    und   pdflatex ./LaTeX/main.tex   ausführen.
%%
%%    Genauere Anpassung des Layouts durch Anpassen der Dateien im
%%    ./LaTeX-Ordner möglich.
%%
%%%%%%%%%%%%%%%%%%%%%%%%%%%%%%%%%%%%%%%%%%
%%%%%%%%%%%%%%%%%%%%%%%%%%%%%%%%%%%%%%%%%%


%%%
%% Erstelle Web-Version oder Druck-Version
%%%
\def\version{web} % "druck" oder "web"


%%%
%% Eigenschaften der Verbindung
%%%

% Verbindung
\def\verbindung{Erlangen} % z.B. Erlangen

% Farben
\definecolor{erstefarbe}{rgb}{0 0 0} % schwarz
\definecolor{zweitefarbe}{rgb}{1 1 1} % weiß
\definecolor{drittefarbe}{rgb}{0.92 0.66 0.24} % gold

% Welche Farbe sollen die Zwischenüberschriften haben?
\definecolor{ueberschriftenfarbe}{rgb}{0.66 0.60 0.24} % altgold


%%%
%% Semester
%%%

\def\sommerwinter{sommer} % "sommer" oder "winter"
\def\semesterjahr{2012} % 2011
\def\semesterzweitesjahr{} % 12


%%%
%% Deckseite
%%%

% Großer Verbindungsname auf dem Deckblatt
%   \zirkel  fügt den Zirkel der Verbindung ein.
%   z.B.   "Erlanger \zirkel Wingolf"
\def\verbindungdeckblatt{Erlanger \zirkel Wingolf}

% Band anzeigen?  (z.B. nicht, wenn es schon vorgedruckt ist.)
\def\bandanzeigen{ja} % "ja" oder "nein"


%%%
%% Rückseite
%%%

% Wingolfslogo, Leerseite oder Ergänzungsseite?
\def\rueckseitelogo{ergaenzung}  % "logo" oder "leer" oder "ergaenzung"


%%%
%% Selbstbeschreibung
%%%

\def\selbstbeschreibung{%%
\vspace{}
\Ueberschrift{Wer wir sind}
Der Erlanger Wingolf ist eine Studentenverbindung, die sich unabhängig von Nationalität und Konfession zum christlichen Glauben bekennt. Daher lehnen wir Duell und Mensur seit jeher kategorisch ab.

Auf der Basis von Weltoffenheit, Freundschaft und Verantwortung gegenüber Verbindung und Gesellschaft unterstützen wir uns gegenseitig, und das nicht nur über die Dauer des Studiums, sondern – über Generationen und Fakultäten hinweg – ein Leben lang. Als Zeichen dieser brüderlichen Verbundenheit tragen wir unser schwarz-weiß-goldenes Band.

Unsere Verbindung ist Mitbegründer des Wingolfsbundes, zu dem 35 Verbindungen in Deutschland, Österreich und Estland gehören, und besteht in ihren Grundzügen seit 1850.

Wir treffen uns in der Friedrichstraße 26, sowohl, um unser Studium in gegenseitiger Unterstützung erfolgreich zu bewältigen, als auch zu geselligen Veranstaltungen und Vorträgen aller Art.

Wir sind offen für engagierte Studenten der FAU, die mehr wollen als nur ein Fachstudium.

Wir sind der Überzeugung, dass dieser Freundeskreis ein hervorragender Ausgleich zum Studienalltag ist und freuen uns auf Deinen Besuch!
}%%


%%%
%% Kontakt-Seite
%%%

% Verbindungsname auf der Kontaktseite
\def\verbindungkontaktseite{Christliche Studentenverbindung\\ Erlanger Wingolf}

% Kontaktdaten
\def\kontaktdaten{
Friedrichstraße 24\,-\,26\\
91054 Erlangen

\daten{Kontakt}{
	Tel.	& 09131\,/\,213\,14 \\
	E-Mail & info@erlanger-wingolf.de \\
	Netz & www.erlanger-wingolf.de
}

\daten{Bankverbindung}{
	Kto. & 21089 \\
	BLZ & 763\,500\,00\\
	& Kreissparkasse Erlangen
}
}


%%%
%% Chargierte
%%%

\def\chargentext{
\Ueberschrift{Die Chargierten}
\Charge{Senior (x)}{Heiko Seeburg}
\Charge{Fuxmaiores (xx)}{Theodor Meedt    Fabian Kollhof}
\Charge{Kneipwart (xxx)}{Martin Seeburg}
\Abstand

laden mit diesem Programm herzlich zu den Veranstaltungen im \semestereinzeilig ein.

Gäste sind zu allen Instituten stets willkommen. Es ist zu beachten, dass Convente interne Veranstaltungen sind.

Für weitere inoffizielle Veranstaltungen bitte den Terminkalender im Netz überprüfen:

www.erlanger-wingolf.de}


%%%
%% Kalender
%%%

%  "\newline" erzeugt einen Umbruch

\def\kalenderteil{

\Monat{Oktober 2011}
	\Eintrag{16. & So & 20\hst & Semesterbegrüßungs\-abend & \o}
	\Eintrag{18. & Di & 18\hst & feierliches Fahnehissen & \ho}
	\Eintrag{& & 20\hst & Anfangsgottesdienst St.-Bonifaz-Kirche & \ho}
	\Eintrag{19 & Mi & 20\hst & Anconvente & \ho}

\Monat{November 2011}
	\Eintrag{3. & Do & 20\hst & Cocktailparty & \io}
	\Eintrag{9. & Mi & 19\hct & Vortrag  G.\,Rambach: „Schicksalsjahre in der Oberpfalz: 1933--59“ & \o}
	\Eintrag{11. & Fr & 20\hct & Damenessen & \io}
	\Eintrag{16. & Mi & 20\hct & Convente & \o}

	\Feierlichkeiten{Feierlichkeiten zum 162.\,Stiftungstag}
	\Eintrag{25. & Fr & 20\hct & Begrüßungsabend & \o}
	\Eintrag{26. & Sa & 18\hst & Ernste Feier & \ho}
	\Eintrag{ & & 20\hct & Festcommers & \ho}
	\Eintrag{27. & So & 10\hst & Kirchgang, mit anschl. Frühschoppen & \ho}

\Monat{Dezember 2011}
	\Eintrag{6. & Di & 20\hct & Nikolauskneipe & \o}
	\Eintrag{7. & Mi & 20\hct & Convente & \o}
	\Eintrag{14. & Mi & 20\hct & Krambambuli & \o}
%
	%\Feierlichkeiten{Thomastag 2011}
	\Eintrag{17. & Sa &15\hct & AK Zukunft & \io}
	\Eintrag{& & 20\hct & Thoamskneipe & \ho}
	\Eintrag{18. & So & 10\hst & Kirchgang & \ho}
	\Eintrag{& & 12\hst & Mittagessen}
	\Eintrag{& & 13\hct & Thomasbummel}
	\Eintrag{& & 15\hct & Kaffeetrinken adH.}
	\Eintrag{21. & Mi & 19\hct & Weihnachtsfeier & \io}

\Monat{Jaunar 2012}
	\Eintrag{11. & Mi & 20\hct & AHC und Convente & \ho}
	\Eintrag{13. & Fr & 20\hct & Gentlemen's Club & \io}
	\Eintrag{18. & Mi & 20\hct & Bibelstunde & \o}
	\Eintrag{20. & bis & 22. & Fuxenfahrt & \ho}
	\Eintrag{25. & Mi & 20\hct & Chargenwahlconvent & \ho}

\Monat{Februar 2012}
	\Eintrag{7. & Di & 19\hst & Abschlussgottesdienst & \ho}
	\Eintrag{8. & Mi & 20\hct & Abconvente & \ho}
	\Eintrag{11. & Sa & 18\hst & Ernste Feier & \o}
	\Eintrag{&& 20\hct & Abkneipe mit anschl. Fahneeinholen & \ho}
}

%%%
%% Ergänzungsseite (statt Wingolfsllog auf Rückseite)
%%%

\def\ergaenzungsteil{
\Ueberschrift{Ausblick: WS\,2012/13}
	\Eintrag{16. & So & 20\hst & Semesterbegrüßungs\-abend & \o}
	\Eintrag{18. & Di & 18\hst & feierliches Fahnehissen & \ho}
	\Eintrag{& & 20\hst & Anfangsgottesdienst St.-Bonifaz-Kirche & \ho}
	\Eintrag{19 & Mi & 20\hst & Anconvente & \ho}

}


%%%
%% Weitere Ankündigungen und Legende
%%%

\def\ankuendigungen{
	\Punkt Jeden Sonntag treffen wir uns fakultativ um 9\hct adH. zum gemeinsamen \Fett{Frühstück mit anschließendem Kirchgang}. Es wird abgewechselt zwischen katholischen und evangelischen Gottesdiensten. Gäste sind jederzeit herzlich willkommen.
	\Punkt \Fett{Fuxenstunden} finden wöchentlich nach Absprache mit dem XX statt.
	\Punkt Jeden Mittwoch treffen wir uns um 19\hst zum \Fett{Aktivenabend} \adH
	Aktive, die sich nicht daran beteiligen können, haben bei den Chargierten Dispens einzuholen.
}

% Adresse für die Erklärung von "adH." in der Legende
\def\adresse{Friedrichstraße 24-26, Erlangen}



