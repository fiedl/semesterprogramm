
%%%
%% Farben
%%%

%\def\goldrgb{0.92 0.66 0.24}  % gold
%\def\goldrgb{0.66 0.60 0.24}  % altgold


%%%
%% Text-Stile
%%%

% Achtung Konsistenz: Im Style keine Abstände definieren. Die kommen weiter unten.

\def\textstyle{\rmfamily\small\normalcolor}

\def\chargentextstyle{\textstyle}
\def\kalendertextstyle{\textstyle\footnotesize}
\def\legendenstyle{\textstyle\tiny\color{Gray}}

\def\selbstbeschreibungstextstyle{\textstyle\rmfamily\scriptsize}


%%%
%% Überschriften-Stile
%%%

\def\sectionstyle{\rmfamily\normalsize}
\def\subsectionstyle{\rmfamily\normalsize\color{ueberschriftenfarbe}\centering}


%%%
%% Abstände
%%%

\def\absatzabstand{\setlength{\parskip}{0.5em}\renewcommand{\baselinestretch}{1}}
\def\ueberschriftabstand{\setlength{\parskip}{0.5em}\renewcommand{\baselinestretch}{1}}

\def\chargenabstand{\absatzabstand\setlength{\parskip}{1em}}
\def\kalenderabstand{\renewcommand{\baselinestretch}{0.8}\setlength{\parskip}{0.2em}}



% Section
\makeatletter
\newcommand\Section[1]{{%
\ueberschriftabstand%
\sectionstyle #1\\
\raisebox{0.6em}{%
	\textcolor{ueberschriftenfarbe}{%
		\rule{\columnwidth}{0.4pt}}%
}%
\vspace*{-0.5em}%
\nobreak\par
}}
\makeatother

% Subsection
\newcommand\Subsection[1]{{%
\subsectionstyle #1\par
}}

%%%
%% Tabellen
%%%

\newcommand{\oldarraystretch}{}
\newcolumntype{C}{>{\centering\arraybackslash}X}
\newcolumntype{R}{>{\raggedleft\arraybackslash}X}
\newcolumntype{L}{>{\raggedright\arraybackslash}X}
%\newcolumntype{L}[1]{>{\raggedright\arraybackslash}p{#1}} % linksbündig mit Breitenangabe
%\newcolumntype{C}[1]{>{\centering\arraybackslash}p{#1}} % zentriert mit Breitenangabe
\newcolumntype{P}[1]{>{\raggedleft\arraybackslash}p{#1}} % rechtsbündig mit Breitenangabe

%\renewcommand\tabularxcolumn[1]{m{#1}}

\newenvironment{tabelle}[2]%
{%
\renewcommand{\oldarraystretch}{\arraystretch}
  \renewcommand{\arraystretch}{#2} % Tabellenabstände vergrößern
  \noindent\tabularx{\columnwidth}{#1}
}{%
  \endtabularx
  \renewcommand{\arraystretch}{\oldarraystretch}
}