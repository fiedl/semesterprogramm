
%%%
%% Seiten-Layout
%%%

% Keine Seitenzahlen
\pagestyle{empty}


%%%
%% Spalten-Aufteilung
%%%

% Spalten-Abstand
\setlength{\columnsep}{1cm}

% Spaltenfeld
\def\boxhoehe{0.46\paperheight}
\newcommand\normalespalte[1]{%
%\fbox{%
\begin{minipage}{\columnwidth}
\vbox to \boxhoehe{#1}
\end{minipage}%
%}%
}

% Spaltenfeld, auf dem Kopf
\newcommand\gedrehtespalte[1]{%
\rotatebox{180}{%
\normalespalte{#1}%
}}

% Vierspaltige Umgebung
% Der Parameter sollte 1 (oben anfangen) oder 51 (bei Seitenhälfte anfangen) sein.
\newenvironment{vierspaltig}[1]{%
\begin{textblock}{98}(0,#1)%
\begin{minipage}[t][0.46\paperheight][t]{0.96\paperwidth}%
\begin{multicols*}{4}%
}{%
\end{multicols*}%
\end{minipage}%
\end{textblock}%
}


%%%
%% Absolute Positionierung
%%%

% Textblock-Positionierung
\setlength{\TPHorizModule}{\paperwidth/100}
\setlength{\TPVertModule}{\paperheight/100}
\textblockorigin{0mm}{0mm}


%%%
%% Layout für Druckversion
%%%
\def\layoutdruck{
	\begin{vierspaltig}{1}
		\normalespalte{
			\chargenseite }
	
		\columnbreak
		\kalenderseiten
	
		\normalespalte{
			\ankuendigungsseite }
	\end{vierspaltig}
	\begin{vierspaltig}{51}
		\gedrehtespalte{
			\deckseite }

		\gedrehtespalte{
			\rueckseite }
		
		\gedrehtespalte{
			\selbstbeschreibungsseite }
		
		\gedrehtespalte{ 
			\kontaktseite }
	\end{vierspaltig}
}


%%%
%% Layout für Web-Version
%%%
\def\layoutweb{
	\begin{vierspaltig}{1}
		\normalespalte{
			\deckseite }

		\columnbreak
		\kalenderseiten
	
		\normalespalte{ 
			\kontaktseite }
	\end{vierspaltig}
	\begin{vierspaltig}{51}
		\normalespalte{
			\rueckseite }

		\normalespalte{
			\chargenseite }
		
		\normalespalte{
			\selbstbeschreibungsseite }

		\normalespalte{
			\ankuendigungsseite }
	\end{vierspaltig}
}